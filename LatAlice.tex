\documentclass[14pt]{article}  
\usepackage{ucs} 
\usepackage[utf8x]{inputenc}
\usepackage[russian]{babel}
\def\baselinestretch{1.5}
\begin{document}  
\section*{Standart ML and Alice} 

Для того чтобы понять, что такое язык программирования Alice, нужно узнать о таком языке программирования как SML. Потому что язык Alice является диалектом языка SML.

Standard ML — компилируемый язык программирования общего назначения высшего порядка, основанный на системе типов Хиндли — Милнера.

Отличается математически точным определением (гарантирующим идентичность смысла программ вне зависимости от компилятора и аппаратного обеспечения), имеющим доказанную надёжность статической и динамической семантики. Является «в основном функциональным» языком, то есть поддерживает большинство технических свойств функциональных языков, но также предоставляет развитые возможности императивного программирования при необходимости. Сочетает устойчивость программ, гибкость на уровне динамически типизируемых языков и быстродействие на уровне языка Си; обеспечивает превосходную поддержку как быстрого прототипирования, так и модульности и крупномасштабного программирования.

SML был первым самостоятельным компилируемым языком в семействе ML и до сих пор служит опорным языком в сообществе по развитию ML (successor ML). В SML впервые была реализована уникальная аппликативная система модулей.

Язык изначально ориентирован на крупномасштабное программирование программных систем: он предоставляет эффективные средства абстракции и модульности, обеспечивая высокий коэффициент повторного использования кода, и это делает его подходящим также для быстрого прототипирования программ, в том числе и крупномасштабных. Например, в процессе разработки (тогда ещё экспериментального) компилятора SML/NJ (60 тысяч строк на SML), порой приходилось вносить радикальные изменения в реализации ключевых структур данных, влияющих на десятки модулей — и новая версия компилятора была готова в течение дня. При этом, в отличие многих других языков, подходящих для быстрого прототипирования, SML может очень эффективно компилироваться.

SML известен своим относительно низким порогом вхождения и служит языком обучения программированию во многих университетах мира. Обширно документирован в рабочем виде, активно используется учёными в качестве базы для исследования новых элементов языков программирования и идиом. К настоящему времени все реализации языка (включая устаревшие) стали открытыми и свободными.

Alice — язык функционального программирования, разработанный в лаборатории Programming Systems Lab в Саарском университете. Это диалект языка Standard ML, дополненный ленивыми вычислениями, конкурентностью (многопоточностью и распределёнными вычислениями на основе вызова удалённых процедур) и программированием в ограничениях.
Реализация Alice Саарского университета использует виртуальную машину SEAM (Simple Extensible Abstract Mashine). Она является свободным программным обеспечением и использует компиляцию «на лету» как в байт-код, так и в родной код для архитектуры x86.

Ранние версии Alice работали в виртуальной машине Mozart/Oz, предоставляя возможность взаимодействия кода на Alice и на Oz.

Возможность вызова удалённых процедур в Alice зависит от виртуальной машины, потому что она использует непосредственную пересылку исполняемого кода с одного компьютера на другой.

Alice расширяет Standard ML рядом примитивов для экзотичной модели нестрогих вычислений, носящей название вызов-по-преднамеченности, с помощью которых легко реализуется параллелизм. Потоки могут быть созданы с помощью зарезервированного слова spawn.

\end{document}